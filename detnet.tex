\documentclass[runningheads]{llncs}
\usepackage[T1]{fontenc}
\usepackage[acronyms]{glossaries}
\usepackage{etoolbox}
\usepackage{hyperref}
\usepackage{graphicx}
\usepackage{color}
\usepackage{lipsum}
\renewcommand\UrlFont{\color{blue}\rmfamily}

% Redefine the glossary section style to match llncs
\makeatletter
\renewcommand{\glossarysection}[2][]{
  \section*{#2}
  \phantomsection
  \def\@glossarysection{#2}
}
\makeatother

\makenoidxglossaries{}
\loadglsentries{abbr.tex}

\begin{document}

\title{Mastering the Clock: The Emergence of Deterministic Networking}
\author{Alexander Hurbean\inst{1} \and
  Pantelis A. Frangoudis\inst{1}\orcidID{0000-0001-6901-7714}}
\authorrunning{Hurbean et al.}
\institute{Technical University of Vienna, 1040 Vienna, Austria}

\maketitle              % typeset the header of the contribution

\begin{abstract}
  The abstract should briefly summarize the contents of the paper in
  150--250 words.
  
  \keywords{real time, deterministic networking, time sensitive networking}
\end{abstract}

\section{Introduction}

As technology, the Internet, and industries rapidly evolve the demand for real-time applications increase and new networking challenges arise. Traditional best-effort packet delivery offered by standard Ethernet networks may not adequately meet the needs of real-time applications, such as the upcoming Industry 4.0 automation, particularly in the \gls{iot} and \gls{cps}~\cite{Wollschlaeger2017}. For instance, industrial networks began to expand beyond the confines of local area networks, necessitating deterministic characteristics across wide-area networks, such as those connecting multiple factories or linking a factory to cloud-based systems. Another issue is the various technologies currently used in industrial networks such as \gls{hdmi}, and \gls{can}, which are proprietary, not interoperable, and require dedicated hardware. \gls{detnet} emerges as a key solution, providing guaranteed latency and reliability for packet data transmission even over multi-hop paths and allowing for the integration of different technologies over packet networks.

The \gls{detnet} \gls{qos} focuses on worst-case values for end-to-end latency, jitter, and packet misordering, as these values impact real-time system performance. \gls{detnet} employs three techniques, as outlined in RFC 8655, to provide these \gls{qos} guarantees:

\begin{itemize}
  \item \textbf{Resource allocation}: Assigns resources such as buffer space and link bandwidth to \gls{detnet} flows, addressing latency and packet loss requirements.
  \item \textbf{Service protection}: Mitigates packet loss due to random media errors and equipment failures through mechanisms like packet replication and elimination or packet encoding. This technique distributes \gls{detnet} flows across multiple paths in time and/or space to minimize packet loss.
  \item \textbf{Explicit routes}: Provides pre-determined paths for data packets to minimize network congestion and deliver predictable performance for time-sensitive applications.
\end{itemize}

These techniques can be applied individually or in combination, with eight possible combinations, including no \gls{detnet}. Some combinations offer broader utility than others, and the \gls{detnet} architecture is designed to facilitate interoperability with existing and developing standards. Examples of typical combinations include seamless redundancy mechanisms on ring topologies, resource allocation as offered by Audio Video Bridging, and the use of all three techniques together for maximum protection. While simpler methods like prioritization and over-provisioning may be sufficient for some applications, \gls{detnet} provides a more comprehensive solution for critical real-time systems~\cite{rfc8655}.

This paper explores the concept of \gls{detnet}, its importance in overcoming the limitations of best-effort packet delivery, and its relevance in many real-world applications.

The paper begins with a ``background'' section that provides a historical context and discusses the developments of \gls{detnet} and time-sensitive networking. It then examines the constraints of best-effort packet delivery and the need for specific latency and reliability guarantees for certain application classes. Additionally, an overview of proposed IEEE 802.11 Standards is presented.

The paper subsequently provides an overview of different domains where \gls{detnet} plays a vital role, such as industrial automation, cellular operation, and professional multimedia. Within each domain, exemplary \gls{detnet} use cases are presented, showcasing how \gls{detnet} can significantly enhance performance, safety, and efficiency.

Finally, the paper discusses the technical foundations that enable \gls{detnet} to build on best-effort packet networks and offer the required guarantees for latency, reliability, and jitter. Key architectural components, protocols, and enabling technologies are examined, providing a deeper understanding of the technical foundations of \gls{detnet}.

By examining the history, applications, and technologies underpinning \gls{detnet}, this paper aims to provide an overview and understanding of the significance of \gls{detnet}. With the continued growth of the \gls{iot}, \gls{cps}, and Industry 4.0, \gls{detnet}'s importance is poised to expand, shaping the future of connectivity.

\section{Background}

This section provides a deeper historical context of the key developments and technologies that have influenced the emergence of \gls{detnet}.

\subsection*{Audio Video Bridging (AVB) and Time-Sensitive Networking (TSN)}
The origins of \gls{detnet} can be traced back to the development of IEEE \gls{avb} standards, which aimed to provide synchronized, low-latency streaming of audio and video data over Ethernet networks. The need for such a solution stemmed from the fact that larger audio/video systems were increasingly difficult to maintain and required purpose-built hardware. Additionally, \gls{avb} allows normal Ethernet traffic to coexist with the audio/video traffic, which is a key feature for the adoption of \gls{avb} in professional environments~\cite{Lim2012}. The IEEE 802.1 task group responsible for the \gls{avb} standard renamed it to \gls{tsn} after completion to better reflect its extended scope, which includes other types of real-time traffic, such as automotive and industrial control traffic~\cite{Wollschlaeger2017}. Although \gls{avb} seemed very promising for many other applications as well, it suffered from limitations such as only supporting 7 hops and only operating on Layer 2 thus not allowing communications across networks~\cite{Imtiaz2009}.

\subsection*{Industrial Fieldbus and Industrial Ethernet}
In parallel, industrial serial field bus systems, such as ProfiBus or Modbus, and industrial Ethernet, were developed to address the specific needs of automation and control systems in industrial environments since best-effort packet services, like standard Ethernet, were not suitable for such applications. The main issues with solutions such as standard Ethernet networking include high cost and unpredictability of traffic due to collisions and congestion loss~\cite{Finn2018}. Still, these early efforts laid the groundwork for the evolution of \gls{detnet}, paving the way for the advancements we see today. 

\subsection*{Wireless Solutions}
Wireless solutions have also been developed to address the need for mobility and flexibility in industrial settings. However, wireless communication has its own set of challenges, such as limited bandwidth and potential interference. To overcome these challenges, new wireless technologies such as WirelessHART and ISA100 have been developed specifically for industrial automation and control applications. These technologies provide reliable and secure communication with low latency and high data rates, making them ideal for use in critical control systems. Despite the challenges, wireless solutions have become increasingly popular in industrial settings due to their ease of installation and flexibility.

Furthermore, the integration of 5G with TSN and Detnet is expected to bring significant improvements to industrial communication systems. 5G's high data rates, low latency, and reliability make it a promising solution for industrial applications, while TSN and Detnet provide deterministic guarantees for time-critical data transmission. This integration is expected to enable the development of new applications that require real-time communication and coordination between machines, such as collaborative robotics and autonomous vehicles. As such, the future of industrial communication systems looks promising, with a wide range of possibilities for increased efficiency and productivity. For example, factories could use wireless sensors to monitor equipment performance in real-time, allowing for predictive maintenance to reduce downtime and optimise production. Additionally, autonomous vehicles in a warehouse could communicate with each other and with the central control system to optimise their routes and avoid collisions, improving safety and efficiency.

\subsection*{Standardization and RFCs}
Standards concerning \gls{tsn} are already finished and released by the IEEE 802.1 task group, such as IEEE standard 802.1AS or IEEE standard 802.1CB which can be used to aid in building deterministic networks~\cite{Finn2018}.

Many standards or RFCs concerning \gls{detnet} are still in development, but some are already released. The IETF \gls{detnet} working group is currently responsible for developing the internet-drafts for \gls{detnet}, which are works in progress and are not yet standardized. The IETF \gls{detnet} working group is currently working on drafts of the following topics:
\begin{itemize}
  \item \gls{detnet} YANG Model
  \item \gls{detnet} Topology YANG Model
  \item \gls{detnet} Controller Plane Framework
  \item Operations, Administration and Maintenance (OAM) for \gls{detnet} with \gls{ip}/\gls{mpls} Data Plane
  \item Requirements for Scaling \gls{detnet}
\end{itemize}

Some already-released RFCs are for example:
\begin{itemize}
  \item RFC 8578: \gls{detnet} Use Cases~\cite{rfc8578}
  \item RFC 8557: \gls{detnet} Problem Statement~\cite{rfc8557}
  \item RFC 8655: \gls{detnet} Architecture~\cite{rfc8655}
  \item RFC 8939: \gls{detnet} Data Plane: \gls{ip}~\cite{rfc8939}
  \item RFC 8964: \gls{detnet} Data Plane: \gls{mpls}~\cite{rfc8964}
  \item RFC 9023: \gls{detnet} Data Plane: \gls{ip} over IEEE 802.1 \gls{tsn}~\cite{rfc9023}
\end{itemize}

\section{Overcoming the Limitations of Best-Effort Packet Delivery}
The limitations of the Internet's best-effort packet delivery service have been a challenge for applications that require specific latency and reliability guarantees. For instance, industrial control systems, telemedicine, and financial trading require timely and predictable packet delivery. The traditional approach of overprovisioning the network to ensure performance is not scalable and can be prohibitively expensive. Therefore, there is a need for a new networking paradigm that provides deterministic guarantees for specific traffic flows. This is where DetNet comes in. DetNet is an architecture that enables deterministic packet forwarding and delivery in a network.

An example of DetNet in action can be seen in a factory that relies on real-time control systems. With DetNet, critical control packets can be given priority and delivered with low jitter and high reliability, ensuring that the machines on the factory floor are properly automated and operate efficiently. This also allows for better coordination between different machines and processes, making the entire system more responsive and reliable overall.

DetNet achieves this by reserving network resources, such as bandwidth and buffer space, for specific traffic flows and providing strict latency and reliability guarantees. DetNet also supports traffic engineering and congestion control to ensure that network resources are used efficiently. DetNet is particularly useful for applications that require real-time communication, such as industrial automation and multimedia streaming. In addition, DetNet can also improve the performance of traditional best-effort applications by providing more predictable network behaviour. However, implementing DetNet requires changes to the network infrastructure and protocols, which can be challenging and time-consuming.

One of the key challenges of implementing DetNet is ensuring interoperability with existing network protocols and equipment. DetNet requires brand-new protocols and mechanisms for resource reservation, traffic engineering, and congestion control that legacy network devices might not be able to support. Therefore, network operators may need to upgrade their equipment or deploy new hardware that supports DetNet. In addition, DetNet may also require changes to the network architecture, such as the deployment of edge computing and network slicing, to provide end-to-end determinism. Despite these challenges, DetNet has the potential to revolutionise the way we design and operate networks, providing unprecedented levels of reliability and predictability for critical applications. As such, it is an area of active research and development, with ongoing efforts to standardise and commercialise DetNet technologies. While the road ahead may be challenging, the benefits of DetNet are clear, and it is likely to play an increasingly important role in the future of networking.

\section{Real-World Applications of Deterministic Networking}

Deterministic networking has numerous real-world applications across a variety of domains.

\subsection*{Industrial Automation}
One such domain is industrial automation, where deterministic networking can provide highly reliable and predictable communication between machines and sensors. For example, in a manufacturing plant, machines must communicate with each other and with central control systems in a precise and timely manner to ensure efficient and safe operations. Deterministic networking can help to ensure that messages are delivered in a predictable and timely manner, reducing the risk of errors and downtime.

\subsection*{Cellular Networks}
Another domain where deterministic networking can be useful is cellular networks, where it can be used to provide reliable and predictable communication between base stations and mobile devices. For example, in a 5G network, deterministic networking can be used to ensure that critical control messages are delivered with low latency and high reliability, enabling new applications such as autonomous vehicles and remote surgery.
%Another domain where deterministic networking is important is cellular network operation, where it can help ensure reliable and low-latency communication for mission-critical applications such as emergency services and autonomous vehicles. 

\subsection*{Internet of Things}
Deterministic networking can also be useful in the Internet of Things (IoT), where it can be used to provide reliable and predictable communication between IoT devices and cloud services. For example, in a smart home, deterministic networking can be used to ensure that critical control messages are delivered with low latency and high reliability, enabling new applications such as remote monitoring and control.

\subsection*{Multimedia Content Production}
In the field of professional multimedia content production, deterministic networking can ensure that audio and video streams are delivered in sync, resulting in high-quality productions. For example, live broadcasting of sports events requires precise coordination between multiple cameras and audio sources. Deterministic networking can help achieve this synchronisation, resulting in a seamless viewing experience for audiences.

\subsection*{Healthcare Industry}
In the healthcare industry, deterministic networking can be used to ensure reliable and timely communication between medical devices, such as monitors and infusion pumps, and electronic health records. This can help improve patient safety by reducing the risk of errors and delays in treatment. For example, in a hospital, deterministic networking can be used to ensure that critical patient data is delivered with low latency and high reliability, enabling new applications such as remote monitoring and control.

\subsection*{Financial Trading}
For instance, in high-frequency trading, where speed is of the utmost importance, deterministic networking can provide a low-latency communication network that can ensure trades are executed in real-time with minimal risk of errors or delays. Even a small delay could cause significant financial loss for the parties involved. With deterministic networking, traders can make split-second decisions and execute trades instantly, giving them a competitive edge over their competitors.

\section{The Technical Foundations of Deterministic Networking}
Deterministic networking is a technology that provides a predictable and reliable communication network, which is essential for real-time applications such as industrial automation, power grids, and autonomous vehicles. At its core, deterministic networking is based on a time-triggered architecture, where data is transmitted in fixed time slots to ensure that it arrives at its destination on time. This approach contrasts with traditional networking, where data is transmitted on a best-effort basis, and the time it takes to reach its destination is unpredictable. To achieve deterministic behavior, deterministic networking uses protocols that prioritize traffic, ensure timely delivery, and minimize packet loss.
One of the key enabling technologies for deterministic networking is TSN, which is a set of IEEE 802.1 standards that provide deterministic guarantees for Ethernet-based networks. TSN offers a range of features, such as time synchronisation, traffic shaping, and path redundancy, that enable deterministic behaviour in the network. By using TSN, deterministic networking can provide low-latency, high-reliability communication for a variety of applications. Another important aspect of deterministic networking is its ability to coexist with legacy networks, allowing for a gradual migration towards deterministic behaviour.

\subsection*{Clock Syncronization}
One of the fundamental requirements for deterministic networking is clock synchronisation. In a time-triggered network, all devices must have a common notion of time to ensure that data is transmitted at the correct time. This requires accurate clock synchronisation across the network, which can be achieved using protocols such as Precision Time Protocol (PTP) or Time-Sensitive Networking (TSN). PTP is a widely adopted standard for clock synchronisation in industrial automation and power grids, while TSN offers enhanced clock synchronisation capabilities for Ethernet-based networks. By ensuring accurate clock synchronisation, deterministic networking can provide reliable and predictable communication for real-time applications.

\subsection*{Time Aware Shaper (IEEE 802.1Qbv)}
Another key technology for deterministic networking is the Time Aware Shaper (TAS), which is a traffic shaping mechanism that ensures that data is transmitted at the correct time. In a time-triggered network, data must be transmitted in fixed time slots to ensure that it arrives at its destination on time. TAS enforces this requirement by shaping traffic according to a predefined schedule, which is determined by the network topology and traffic patterns. By using TAS, deterministic networking can provide low-latency, high-reliability communication for a variety of applications.
%TAS is a key TSN feature that enables traffic shaping in a deterministic network. It allows the network to allocate bandwidth to different types of traffic based on their priority and timing requirements. This ensures that critical traffic, such as control messages and real-time data, is given priority over less time-sensitive traffic, such as email and file transfers. The TAS also ensures that traffic is transmitted within its designated time slot, preventing delays and jitter that can affect real-time applications. 

\subsection*{Frame Preemption (IEEE 802.1Qbu)}
Another important TSN feature is frame preemption, which allows high-priority traffic to interrupt and preempt lower-priority traffic in real-time. This ensures that critical messages are delivered without delay, even in congested network conditions. Frame Preemption works by inserting a special message, called a preemption frame, into the network that signals the lower-priority traffic to pause transmission and allow the high-priority traffic to be transmitted immediately. This feature is particularly useful in industrial automation and control systems, where real-time communication is essential for safety and efficiency.

\subsection*{Frame Replication and Elimination for Reliability (IEEE 802.1CB)}
Frame Replication and Elimination for Reliability (FRER) is a TSN feature that provides redundancy for critical traffic. It works by replicating critical messages and sending them over multiple paths to ensure that at least one copy reaches its destination. If a message is lost or corrupted on one path, it can be recovered from another path. This feature is particularly useful in industrial automation and control systems, where reliability is essential for safety and efficiency.
%Another TSN feature that enhances reliability is Frame Replication and Elimination for Reliability (FRER). FRER ensures that critical data is transmitted without loss or corruption by replicating frames and eliminating any duplicates. This is achieved by sending multiple copies of the same frame over different paths in the network and using a voting mechanism to select the correct frame at the receiving end. FRER can be especially useful in applications such as power distribution and transportation systems, where data loss or corruption can have severe consequences.

\section{Conclusion}
In conclusion, deterministic networking is a technology that provides predictable and reliable communication networks, which are essential for real-time applications. It is based on a time-triggered architecture and uses protocols that prioritise traffic, ensure timely delivery, and minimise packet loss. Time-Sensitive Networking (TSN) is a key enabling technology that offers deterministic guarantees for Ethernet-based networks. The potential impact of deterministic networking is significant in various applications and industries such as industrial automation, power grids, and autonomous vehicles. With its ability to coexist with legacy networks, deterministic networking can provide a gradual migration towards deterministic behaviour. Overall, deterministic networking is a promising technology that can revolutionise the way we communicate and operate in real-time applications, and its impact is expected to grow as more industries adopt it.

\printnoidxglossary[type=acronym,sort=letter,title=Abbreviations]

\bibliographystyle{splncs04}
\bibliography{export}
\end{document}
